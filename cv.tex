\documentclass[letterpaper,10pt]{article}

\usepackage{makecell}
\usepackage[link=off]{phonenumbers}

\usepackage{latexsym}
\usepackage[empty]{fullpage}
\usepackage{titlesec}
\usepackage{marvosym}
\usepackage[usenames,dvipsnames]{color}
\usepackage{verbatim}
\usepackage{enumitem}
\usepackage[pdftex]{hyperref}
\usepackage{fancyhdr}

\usepackage[utf8x]{inputenc}
\usepackage[english,russian]{babel}
\usepackage{cmap}

\pagestyle{fancy}
\fancyhf{} % clear all header and footer fields
\fancyfoot{}
\renewcommand{\headrulewidth}{0pt}
\renewcommand{\footrulewidth}{0pt}
\usepackage[margin=0.3in]{geometry}
% Adjust margins
\addtolength{\oddsidemargin}{-0.0in}
\addtolength{\evensidemargin}{-0.0in}
\addtolength{\textwidth}{0in}
\addtolength{\topmargin}{20pt}
\addtolength{\textheight}{0.0in}

\urlstyle{same}

\usepackage{xcolor}% http://ctan.org/pkg/xcolor
\usepackage{hyperref}% http://ctan.org/pkg/hyperref
\hypersetup{
  colorlinks=true,
  linkcolor=blue!50!red,
  linkbordercolor=red,
  urlcolor=blue!70!black
}

\raggedbottom
\raggedright
\setlength{\tabcolsep}{0in}

% Sections formatting
\titleformat{\section}{
  \vspace{-10pt}\scshape\raggedright\large
}{}{0em}{}[\color{black}\titlerule \vspace{-7pt}]

%-------------------------
% Custom commands
\def \ifempty#1{\def\temp{#1} \ifx\temp\empty }

\newcommand{\resumeItem}[2]{
  \item\small{
  	\ifempty{#1}#2\else\textbf{#1}{: #2 \vspace{-2pt}}\fi
  }
}

\usepackage[dvipsnames]{xcolor}
\definecolor{mygray}{gray}{0}
\usepackage{fancybox}

\usepackage{lmodern}
\usepackage{tikz}

% Style definition
\tikzset{rndblock/.style={rounded corners,rectangle,draw,outer sep=0pt}}

% Command Definition
\newcommand{\tframed}[2][]{\tikz[baseline=(h.base)]\node[rndblock,#1] (h) {\color{black}{#2}};}

\newcommand*{\mystrut}{\rule[-0.2\baselineskip]{0pt}{0.8\baselineskip}}
\newcommand{\skill}[1]{\tframed[lightgray]{\mystrut#1}}


\newcommand{\resumeSubheading}[4]{
  \vspace{-1pt}\item
    \begin{tabular*}{0.97\textwidth}{l@{\extracolsep{\fill}}r}
      \textbf{#1} & \textcolor{mygray}{#2} \\
      \textit{\small#3} & \textcolor{mygray}{\textit{\small #4}} \\
    \end{tabular*}\vspace{-5pt}
}

\newcommand{\resumeExpSubheading}[5]{
  \vspace{-1pt}\item
    \begin{tabular*}{0.97\textwidth}{l@{\extracolsep{\fill}}r}
      \textbf{#1}  & \textcolor{mygray}{#2} \\
      \textit{\small#3} & \textcolor{mygray}{\textit{\small #4}} \\
      {\scriptsize#5}
    \end{tabular*}\vspace{4pt}
}

\newcommand{\resumeProjSubheading}[4]{
  \vspace{-1pt}\item
    \begin{tabular*}{0.97\textwidth}{l@{\extracolsep{\fill}}r}
      \textbf{#1}  & \textcolor{mygray}{#2} \\
      \scriptsize {#3} & \textcolor{mygray}{\textit{\small #4}} \\
    \end{tabular*}\vspace{4pt}
}

\newcommand{\resumeSubItem}[2]{\resumeItem{#1}{#2}\vspace{-4pt}}

\renewcommand{\labelitemii}{$\circ$}

\newcommand{\resumeSubHeadingListStart}{\begin{itemize}[leftmargin=*]}
\newcommand{\resumeSubHeadingListEnd}{\end{itemize}}
\newcommand{\resumeItemListStart}{\begin{itemize}[leftmargin=0.2in]}
\newcommand{\resumeItemListEnd}{\end{itemize}\vspace{-5pt}}

\usepackage{changepage}
\newcommand{\resumeDesc}[1]{\begin{adjustwidth}{5pt}{0pt}\vspace{-2pt}{\small{#1}}\end{adjustwidth}}

\begin{document}

%------------HEADING--------------
\begin{tabular*}{\textwidth}{l@{\extracolsep{\fill}}r}
  \textbf{\Large Никита Анхимов} & Email : \href{mailto:anhimovn1@gmail.com}{anhimovn1@gmail.com}\\
  GitHub: \href{https://github.com/nickAnhel}{nickAnhel} & Telegram : \href{https://t.me/nickAnhel}{nickAnhel} \\
\end{tabular*}


%-----------ABOUT-----------------
\section{О себе}
 \resumeSubHeadingListStart
   \item{
     {Начинающий Python-разработчик.  Больше всего заинтересован в создании Web-приложении с использованием фреймворка FastAPI. Также имею небольшой опыт создания Telegram-ботов с помощью библиотеки Aiogram и написания фронтенда на ReactJS.}
   }
 \resumeSubHeadingListEnd


%-----------EDUCATION--------------
\section{Образование}
  \resumeSubHeadingListStart
    \resumeSubheading
       {Прикладная информатика, СПбГУАП}{Санкт-Петербург, Россия}
      {Неоконченный бакалавриат. Институт информационных технологий и программирования}{Сентябрь 2022 - наст. время}
  \resumeSubHeadingListEnd


%-----------EXPERIENCE-------------
\section{Опыт работы}
  \resumeSubHeadingListStart
      \resumeExpSubheading
      {\href{https://algoritmika.org/ru}{Алгоритмика}}{Санкт-Петербург, Россия}
      {Преподаватель}{Декабрь 2022 - Май 2024}
      {\skill{Python} \skill{C#} \skill{JS} \skill{HTML} \skill{CSS}}
      \resumeDesc{
      \begin{itemize}
          \item Занимался обучением детей и подростков от 6 до 17 лет по таким курсам, как «Программирование на Python», «Фронтенд Разработка», «Разработка игр на Unity», «Математика» и другие. 
          \item Проводил открытые уроки и мастер-классы, направленные на привлечение клиентов и популяризацию программирования среди детей и подростков.
      \end{itemize}}
  \resumeSubHeadingListEnd


%-----------PROJECTS--------------
\section{Проекты}
  \resumeSubHeadingListStart
  
    \resumeProjSubheading
      {\href{https://github.com/nickAnhel/FastAPI-Video-Hosting}{Video Hosting} }{}
      {\skill{Python} \skill{FastAPI} \skill{Celery} \skill{PostgreSQL} \skill{RabbitMQ} \skill{S3} \skill{Docker} \skill{JWT}}{}
          \resumeDesc{
          Разработал бэкенд платформы, где пользователи могут публиковать, просматривать и оценивать видео, писать комментарии и создавать плейлисты. Для создания REST API используется фреймворк FastAPI в связке с Pydantic для валидации данных и SQLAlchemy для обращения к БД. Для хранения медиа-файлов (фото профиля, видео, превью) подключено объектное хранилище Selectel S3. Для аутентификации и авторизации пользователей применяется технология JWT.
          
          Также в проекте реализована система отправки уведомлений, которая позволяет пользователям узнавать о выходе новых видео на самой странице видео-хостинга, через электронную почту или Telegram. Эта система вынесена в качестве отдельного сервиса, взаимодействие с которым осуществляется за счет публикации основным бэкендом приложения задач в очередь RabbitMQ. Внутри самого сервиса задачи попадают в Celery, который уже занимается отправкой уведомлений. 
        
          Также я отвечал за деплой приложения на домашнем сервере под управлением EndeavourOS. Для запуска проекта использовался Docker и Docker Compose. Из-за невозможности получения статического IP для сервера был арендован VPS, с которым (с помощью autossh) было реализовано туннельное соединение с пробросом портов для бэкенда и фронтенда приложения.
          }
  
    \resumeProjSubheading
      {\href{https://github.com/nickAnhel/FastAPI-Atom-Chat}{Atom Chat}}{}
      {\skill{Python} \skill{FastAPI} \skill{ReactJS} \skill{SocketIO} \skill{Websockets} \skill{PostgreSQL} \skill{JWT}}{}
        \resumeDesc{
            Разработал мессенджер, где пользователи могут общаться в приватных и публичных чатах, создавать сообщества и управлять историей сообщений.

            Бэкенд приложения реализован с помощью асинхронного фреймворка FastAPI, для валидации данных используется Pydantic, обращение к БД осуществляется с помощью SQLAlchemy и ее Core синтаксиса. Для управления Websocket-соединениями используется мультиплатформенная библиотека SocketIO. Для аутентификации и авторизации пользователей используется технология JWT. Пакет, отвечающий за операции с аккаунтами пользователей, покрыт тестами, для чего применяется фреймворк PyTest.
            
            Фронтенд приложения реализован с помощью библиотеки ReactJS, а для управления Websocket-соединениями применяется библиотека SocketIO.
        }
  \resumeSubHeadingListEnd


%--------ADDITIONAL INFO----------
\section{Дополнительная информация}
 \resumeSubHeadingListStart
   \item{
     { Имею небольшой опыт написания проектов на C++, преимущественно консольных приложений. }
   }\vspace{-7pt}
   \item{
     { С июня 2024 года перешел на Linux и с тех пор расширяю свои познания об этом семействе операционных систем. Использую Fedora как основную ОС на ноутбуке и Arch на домашнем сервере. }
   }\vspace{-7pt}
   \item{
     { В университете изучаю Data Science и ML с использованием библиотек pandas, scikit-learn и numpy, а также внутреннее устройство компьютерных систем и сетей. }
   }
 \resumeSubHeadingListEnd


%--------PROGRAMMING SKILLS---------
\section{Навыки}
 \resumeSubHeadingListStart
   \item{
     \textbf{Языки программирования}{: Python, JavaScript }
   }\vspace{-7pt}
   \item{
     \textbf{Фреймворки и библиотеки}{: FastAPI, SQLAlchemy, Pydantic }
   }
   \vspace{-7pt}
   \item{
     \textbf{Технологии}{: PostgreSQL, Git, Docker  }
   }
   \vspace{-7pt}
   \item{
     \textbf{Операционные системы}{: Windows, Linux (Fedora, Arch, Ubuntu) }
   }
 \resumeSubHeadingListEnd


\end{document}
